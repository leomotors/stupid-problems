\documentclass[11pt,a4paper]{article}

\usepackage{xeCJK}
\usepackage{res/style_th}
\setCJKmainfont{IPAGothic}

\begin{document}

\begin{problem}{โรงคาเฟ่แห่งนภา}{standard input}{standard output}{1 seconds}{256 megabytes}

``มาหมุนซักรอบกัน เอ้า แดนซ์ แดนซ์
ณ โรงคาเฟ่แห่งนภา แดนซ์ แดนซ์
ประตูอยู่ทางไหน? (ทางนี้!)
เข้ามาในถ้วยเลย (กระโจนเข้าไป!)"
(เนื้อเพลงส่วนหนึ่งจากเพลง 天空カフェテリア (Tenkuu Cafeteria) แปลไทยโดย Muse Thailand)

อาโอยามะ บลูเมาท์เท่น นักแต่งนิยายชื่อดัง ผู้มีผลงานดังอย่าง จอมโจรลาแปง, บาริสต้าที่กลายเป็นกระต่าย เป็นต้น อาโอยามะเองก็เป็นผู้ชื่นชอบกาแฟด้วย และต้องการลิ้มรสกาแฟที่หลากหลาย โดยในเมืองแสนสงบแห่งนี้ได้มีร้านกาแฟแห่งหนึ่งที่มีชื่อว่า Rabbit House ซึ่งดูแลโดยหลานสาวของมาสเตอร์ คาฟู ชิโนะ ที่ร้านแห่งนี้มีกาแฟที่หลากหลายและขึ้นชื่อ โดยมีกาแฟถึง $C$ รสด้วยกัน

โดยกาแฟแต่ละรสนั้น เมื่อดื่มแล้วจะทำให้ผู้ดื่มมีความสุข เราวัดสิ่งนี้ด้วยค่าความสุข $H$ โดยกาแฟรสที่ $i$ (เมื่อ $1 \le i \le C$) จะมีค่าความสุข $H_i$

แต่ด้วยความเก่งกาจของหลานสาวมาสเตอร์ กาแฟแต่ละรสนั้น นอกจากจะมีค่าความสุขของตัวเองแล้ว จะมีความเชื่อมโยงกับกาแฟรสอื่นด้วย นั่นคือหากคุณได้ดื่มกาแฟรสหนึ่งต่อจากกาแฟรสที่เข้ากันได้ จะได้รับค่าความสุขมากกว่าปกติ

อาโอยามะ บลูเมาท์เท่น ได้ทำการศึกษากาแฟทั้ง $C$ รส แล้วพบว่ามีความเชื่อมโยงอย่างว่า $O$ รูปแบบ (โดยที่ $0 \le O \le C*(C-1) $) โดยแต่ละรูปแบบแทนด้วย $(a,b,h)$ หมายถึงหากดื่มกาแฟรส $a$ หลังจากกาแฟรส $b$ แล้วจะได้รับค่าความสุขโบนัสเป็นจำนวน $h$

อาโอยามะ บลูเมาท์เท่น ต้องการลิ้มรสกาแฟให้ครบทุก $C$ รสชาติ โดยจะดื่มกาแฟแต่ละรสเพียงรสละ $1$ แก้วเท่านั้น ซึ่งสังเกตว่าลำดับการดื่มที่แตกต่างกันอาจให้ค่าความสุขรวมที่แตกต่างกัน อาโอยามะ บลูเมาท์เท่น จึงได้ไหว้วานคุณซึ่งเป็นนักคอมพิวเตอร์โอลิมปิกที่เก่งกาจที่สุดในเมืองนี้ ให้ช่วยตอบคำถามที่ว่า ค่าความสุขมากที่สุดที่จะได้รับ มีค่าเท่าไหร่

\InputFile

บรรทัดแรกมี $2$ จำนวน ได้แก่ $C$ และ $O$ แทนจำนวนรสกาแฟ และจำนวนรูปแบบความเชื่อมโยงของกาแฟ

บรรทัดที่สอง มี $C$ จำนวน แทนค่าความสุขของกาแฟรสที่ $i$ (เมื่อ $1 \le i \le C$)

บรรทัดที่ $2 + i$ มี $3$ จำนวน ได้แก่ $a_i, b_i, h_i$ แทนรูปแบบความเชื่อมโยงที่ $i$

รับประกันว่ารูปแบบของกาแฟใดๆจะมีค่า $h$ เพียงค่าเดียว

\OutputFile

$1$ จำนวน แสดงผลรวมค่าความสุขที่มากที่สุด

\end{problem}

\end{document}
