\documentclass[11pt,a4paper]{article}

\usepackage{res/style_th}

\begin{document}

\begin{problem}{ขี้เกียจอ่านหนังสือสอบ}{standard input}{standard output}{2 seconds}{512 megabytes}

หลังจากการเรียนออนไลน์อย่างยาวนานกว่า 2 ปี ชีวิต ม.ปลาย ที่สูญเสียไปไม่มีทางกลับมาได้อีกแล้ว

แต่เราย้อนอดีตไม่ได้ มีแต่เพียงต้องมุ่งหน้าต่อไป

\textit{ยินดีต้อนรับเข้าสู่รั้วมหาลัย!!!}

ท่านโอได้เข้าสู่รั้วมหาลัยที่มีสิ่งมากมายให้ค้นหา ทั้งนี้ทั้งนั้น
ในช่วงการเรียนออนไลน์อย่างน่าเบื่อเมื่อ 2 ปีที่ผ่านมา ท่านโอก็ได้ทำสิ่งที่หลายคนทำ \textit{นั่นคือโยน!}
เขาได้ใช้เวลาพวกนั้นไปกับการศึกษา Full Stack Website Development, Application Development,
Cloud Services \& Computing, การเขียน Discord Bot, Data Structures and Algorithms,
Quantum Computing ฯลฯ

และในตอนนี้ท่านโอก็ได้เข้าไปในรั้วมหาลัย และพบเจอกับ \textbf{บอสใหญ่ 3 ตัว} อย่าง ฟิสิกส์ เคมี และ แคล(คูลัส)

\begin{center}
\includegraphics[width=10cm]{./res/lazystudy/big3subj.jpg}
\end{center}

นี่ก็เป็นเวลานานหลายเดือนแล้ว หลังจากที่ท่านโอได้แตะวิชาเหล่านี้
ความจริงแล้ว ท่านโอเทพและถนัดวิชาเหล่านี้มาก เขาเป็นตัวเก็งท็อป ๆ ของรุ่น
หากเขาตั้งใจเรียนแล้วล่ะก็ เขาสามารถทำข้อสอบได้ง่าย ๆ เลย
แต่ประเด็นไม่ใช่อะไรไกลนอกจากคำว่า \textbf{ขี้เกียจ!!!} เพราะเขาเองก็อยากนำเวลาอันมีค่า
ไปศึกษา Quantum Programming ฯลฯ

ในการสอบครั้งหน้านั้น มีคะแนนเต็ม $F$ และมีเนื้อหาในวิชา $N$ บท ซึ่งบางบทอาจจะไม่ออกสอบ
ทางอาจารย์ได้แจ้งมาว่าแต่ละบทจะออกสอบ $S_i$ คะแนน
$(1 \le i \le N,\ 0 \le S_i \le F,\ \sum_{i=1}^{N} S_i = F)$

สังเกตได้ว่าเราไม่จำเป็นต้องรู้ทุกบทเพื่อที่จะสอบผ่าน สิ่งที่ท่านโอต้องการคือ
การผ่านได้เกรด 4 แค่นั้น จะคาบเส้นยังไงก็ได้ ขอเลข 4 สวย ๆ พอ
เพื่อการนั้นในการสอบครั้งนี้ ท่านโอต้องทำให้ได้อย่างน้อย $P$ คะแนน
เห็นเช่นนั้นแล้วท่านโอจึงเริ่มคิดที่จะวางแผนว่าจะ\textit{ฟิต}บทไหนดี

ในบทเรียน $N$ บทนั้น แต่ละบทจะใช้เวลา $T_i$ ในการเรียน ซึ่งเมื่อเรียนจบแล้ว
ด้วยความเก่งกาจของท่านโอ เขาสามารถทำข้อสอบของบทนั้นได้ทั้งหมด
นั่นคือหากเขาเรียนบทที่ $i$ แล้ว เขาจะได้ $S_i$ คะแนน
ทั้งนี้แต่ละบทมีความจำเป็นที่จะต้องใช้ความรู้จากบทก่อน ๆ เรียกว่าเซต $D_i$
ซึ่งจะแทนบทที่ต้องเรียนมาก่อนถึงจะเรียนบทนี้ได้เข้าใจ

ด้วยความซับซ้อนขนาดนี้ ท่านโอจึงขี้เกียจคิดแล้วฝากให้คุณซึ่งมีแววจะเป็นผู้แทนประเทศคนต่อไป
ช่วยหาคำตอบหน่อยว่าจะต้องใช้เวลาเรียนอย่าง\textbf{น้อยที่สุด}เท่าไหร่
จึงจะมีคะแนนอย่างน้อย $P$ คะแนน

\section*{รายละเอียดการเขียนโปรแกรม}

คุณจะต้องเขียนฟังก์ชันต่อไปนี้

\begin{verbatim}
int64_t get_min_study_time(int N, int64_t P, int64_t F,
                           std::vector<int> S, std::vector<int> T,
                           std::vector<std::set<int>> D);
\end{verbatim}

ฟังก์ชันนี้จะรับค่า $N, P, F$ และ $S[i]$ แทนคะแนนที่ออกสอบของบทที่ $i$,
$T[i]$ แทนเวลาที่ต้องใช้เพื่อเรียนบทที่ $i$ และ $D[i]$
แทนเซตของบทที่ต้องเรียนมาก่อน ถึงจะเรียนบทที่ $i$ ได้

\textbf{หมายเหตุ}: vector $S, T$ และ $D$ มีอินเด็กซ์เริ่มต้นที่ $1$
นั่นคือข้อมูลที่อินเด็กซ์ $0$ ไม่เกี่ยวข้องกับข้อมูลทดสอบ

\Scoring

การให้คะแนนในข้อนี้ จะใช้\textbf{กฏเดียวกับ IOI และค่ายสสวท}
นั่นคือผลรวมของคะแนนที่\textbf{ดีที่สุด}ของแต่ละชุดทดสอบย่อยจากทุกการส่ง และ\textbf{ทุกชุดทดสอบย่อยมีการจัดข้อมูลทดสอบแบบกลุ่ม}

\Constraints

\begin{itemize}
\item $1 \le N \le 200\ 000$
\item $1 \le T_i \le 10^{9},\ 0 \le S_i \le 10^{9},\ \sum_{i=1}^{N} S_i = F,\ 1 \le P \le F \le 10^{15}$
\item รับประกันว่าไม่มี Circular Dependency
\end{itemize}

\Subtasks

\textbf{ปัญหาย่อยที่ 1} (1 คะแนน) $N \le 10$ และ $D_i = \{\}\ \forall i$

\textbf{ปัญหาย่อยที่ 2} (3 คะแนน) $N \le 20$

\textbf{ปัญหาย่อยที่ 3} (5 คะแนน) $N \le 100,\ S_i, T_i \le 100$ และ $D_i = \{\}\ \forall i$

\textbf{ปัญหาย่อยที่ 4} (7 คะแนน) $N \le 1\ 000$ และลักษณะของ Dependency เป็นต้นไม้ Star นั่นคือ $D_1 = \{\}$ และ $D_i = \{1\}\ \forall i \ge 2$

\textbf{ปัญหาย่อยที่ 5} (7 คะแนน) $N \le 1\ 000$ และลักษณะของ Dependency เป็นแบบเส้นตรง นั่นคือ $D_1 = \{\}$ และ $D_i = \{i-1\}\ \forall i \ge 2$

\textbf{ปัญหาย่อยที่ 6} (24 คะแนน) $N \le 10\ 000$

\textbf{ปัญหาย่อยที่ 7} (16 คะแนน) ลักษณะของ Dependency เป็นต้นไม้ Star นั่นคือ $D_1 = \{\}$ และ $D_i = \{1\}\ \forall i \ge 2$

\textbf{ปัญหาย่อยที่ 8} (16 คะแนน) ลักษณะของ Dependency เป็นแบบเส้นตรง นั่นคือ $D_1 = \{\}$ และ $D_i = \{i-1\}\ \forall i \ge 2$

\textbf{ปัญหาย่อยที่ 9} (21 คะแนน) ไม่มีเงื่อนไขเพิ่มเติม

\section*{เกรดเดอร์ตัวอย่าง}

เกรดเดอร์ตัวอย่างที่อยู่ในไฟล์แนบจะรับและส่งออกข้อมูลดังนี้

\textbf{ข้อมูลนำเข้า}

บรรทัดแรกประกอบด้วย $N, P$ และ $F$

บรรทัดที่สองประกอบด้วยจำนวนเต็ม $N$ ตัว แทนค่า $S_i$

บรรทัดที่สามประกอบด้วยจำนวนเต็ม $N$ ตัว แทนค่า $T_i$

บรรทัดที่ $3+i$ ถึง $3+N$ ประกอบด้วย $n(D_i)$ และจำนวนเต็ม $n(D_i)$ ตัว แทนเซตของ $D_i$

\textbf{ข้อมูลส่งออก}

เกรดเดอร์ตัวอย่างจะส่งออกจำนวนเต็มหนึ่งจำนวนเป็นค่าที่ถูกรีเทิร์นจาก get\_min\_study\_time

\section*{ตัวอย่างข้อมูลนำเข้าและข้อมูลส่งออกสำหรับเกรดเดอร์ตัวอย่าง}

\begin{example}
\exmp{
5 12 20
6 4 1 5 4
8 6 7 4 3
0
0
1 1
1 1
2 1 3
}{
18
}\exmp{
5 16 20
6 4 1 5 4
8 6 7 4 3
0
0
1 1
1 1
2 1 3
}{
22
}%
\end{example}

\end{problem}

\end{document}
