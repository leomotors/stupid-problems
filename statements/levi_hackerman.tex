\documentclass[11pt,a4paper]{article}

\usepackage{res/style_th}

\begin{document}

\begin{problem}{รีไวล์ แฮกเกอร์แมน}{standard input}{standard output}{1 seconds}{256 megabytes}

รีไวล์ แฮกเกอร์แมน เป็นญาติห่างๆ ของรีไวล์ แอคเคอร์แมน ซึ่งอาศัยอยู่ในโลกคู่ขนานที่เต็มไปด้วยไททัน

รีไวล์ (แอคเคอร์แมน) มีความเคียดแค้นต่อไททันสัตว์ป่าใจเกเร และต้องการฆ่ามันทิ้งเป็นอย่างมาก และโอกาสนี้ก็มาถึงในสนามรบขนาด $N * N$ (อีกแล้ว) ($1\le N\le 10^9$) ซึ่งมีพิกัดตั้งแต่ $1*1$ ถึง $N*N$ โดยรีไวล์อยู่ที่ตำแหน่ง $L(x,y)$ และไททันสัตว์ป่าใจเกเรอยู่ที่ตำแหน่ง $K(x,y)$

แน่นอนว่าการจะเคลื่อนย้ายนั้นก็ต้องใช้เครื่องเคลื่อนย้ายสามมิติ แต่ดูเหมือนว่าอุปกรณ์นี้จะมีข้อจำกัด โดยมีระยะในการเคลื่อนที่หนึ่งครั้งไม่เกิน $D$ ซึ่งจะถือว่าสามารถเดินทางจากจุด $A(x_1,y_1)$ ไป $B(x_2,y_2)$ ได้ก็ต่อเมื่อ $(\Delta x)^2 + (\Delta y)^2 \le A^2$

เนื่องจากสมรภูมินี้มันเถื่อน การที่รีไวล์ (แอคเคอร์แมน) จะเคลื่อนที่นั้น ย่อมไม่อยากแตะพื้นเป็นอันขาด โชคดีที่มีไททันโง่ๆยืนอยู่ทั้งหมด $T$ ตัว (ไม่มีไททันมากกว่าหนึ่งตัวอยู่ในตำแหน่งเดียวกัน รวมถึงรีไวล์และไททันสัตว์ป่าใจเกเร) รีไวล์สามารถใช้วิธีกระโดดตามไททันไปเรื่อยๆ จนถึงตัวของไททันสัตว์ป่าใจเกเร

เนื่องจากจำนวนของไททันมีขนาดมาก ระดับ $T \le 5000$ สมองของคนเตี้ยอย่างรีไวล์ไม่สามารถหาคำตอบได้ จึงได้ไหว้วานคุณ รีไวล์ แฮกเกอร์แมน ญาติห่างๆ ในโลกคู่ขนานที่มี CPU สุดแรงอย่าง AMD Ryzen ให้ช่วยตอบคำถามรีไวล์หน่อยว่าจำนวนครั้งที่น้อยที่สุดที่ต้องใช้ในการกระโดดเคลื่อนย้ายเพื่อไปถึงตัวไททันสัตว์ป่าใจเกเรมีค่าเท่าใด

\InputFile

บรรทัดแรก มี $3$ จำนวนได้แก่ $N$, $D$ และ $T$ แทนขนาดตาราง, ระยะทำการของเครื่องเคลื่อนย้ายสามมิติ และจำนวนไททันโง่ๆ

บรรทัดที่สอง มี $4$ จำนวนได้แก่ $x_L$, $y_L$, $x_K$ และ $y_K$ แทนพิกัดของรีไวล์และไททันสัตว์ป่าใจเกเร

บรรทัดที่ $2+i$ ถึง $2+T$ ($1\le i\le T$) มี $2$ จำนวนได้แก่ $x_i$ และ $y_i$ แสดงพิกัดของไททันโง่ๆ ตัวที่ $i$

\OutputFile

$1$ จำนวน แสดงจำนวนครั้งที่น้อยที่สุดที่รีไวล์ต้องกระโดดเพื่อไปหาไททันสัตว์ป่าใจเกเร

\end{problem}

\end{document}
