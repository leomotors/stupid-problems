\documentclass[11pt,a4paper]{article}

\usepackage{res/style_th}

\begin{document}

\begin{problem}{สนธยา}{standard input}{standard output}{1 seconds}{256 megabytes}

สนธยา \textbf{Code Name} - Twilight

สนธยาเป็นสายลับมือฉมังจากฮิซิรุ เขาเกิดมาในสนามรบ และเห็นเรื่องที่น่าเศร้าแบบนี้ทุกวัน 
ปัจจุบันเป็นบุคคลที่มีความสำคัญต่อความสงบสุขของโลกไปนี้ เขาเป็นสายลับที่ทำหน้านี่ป้องกันอาชญากรรมต่างๆ
เป้าหมายของเขาคือต้องการสร้างโลกที่เด็กไม่ต้องร้องไห้ ไม่อยากให้สิ่งที่เกิดกับเขาในวัยเด็ก

ภารกิจที่เขาต้องทำนั้นมีหลากหลายมาก แต่บ่อยที่สุดคือการสังหารบุคคลสำคัญที่จ้องจะพรากความสงบสุขของโลกไป
สนธยาได้ฉายาว่า พ่อร้อยหน้า นั่นเพราะเขาเคยสวยรอยเป็นคนมานับไม่ถ้วน

ในวันนี้ คุยในฐานะที่เป็นสายให้กับสนธยา ผู้เคยช่วยเหลือสนธยาในการทำภารกิจมาหลายครั้งแล้ว
\st{แต่ไม่เคยได้รับเงินซักครั้ง} กำลังจะมาช่วยสนธยาในการทำภารกิจสำคัญให้มีประสิทธิผลที่ดียิ่งขึ้น

ในภารกิจนี้และอีกหลายๆภารกิจที่ใกล้เคียงกัน คุยจะมีเป้าหมายทั้งหมด $N$ คน โดยงานของคุณนั้นง่ายมาก
คือการเก็บคนเหล่านี้ให้ได้มากที่สุด คุณมีความสามารถในการปลอมตัวเป็นใครก็ได้ แต่ถ้าจะเข้าใกล้คนๆใดได้
คุณก็ต้องปลอมตัวเป็นคนที่คนนั้นไว้ใจ หลังจากที่คุณเก็บใครบางคนไปแล้ว หน้าของคุณจะเปื้อนคราบและจะทำให้
คุณไม่สามารถใช้หน้านั้นได้จนกว่าจะจบภารกิจ

ยกตัวอย่างในภารกิจนี้ คุณได้รับคำสั่งไปจัดการกับผู้ที่ทำลายความสุขของประชาชน ได้แก่
ประยุทธ์ ประวิตร วิษณุ และ ปารีณา \textbf{\textit{(ชื่อสมมุติ ย้ำ ชื่อสมมุติ)}}
แต่ละคนก็จะมีคนที่ไว้ใจต่างกัน โดยประยุทธ์ ประวิตร และ วิษณุ ทุกคนต่างไว้ใจซึ่งกันและกัน
ในขณะที่ปารีณา (ซึ่งเป็นส่วนเกิน) ไม่ถูกไว้ใจโดยใครเลย แต่ปารีณาไว้ใจทุกคน

เมื่อสนธยารู้สถานการณ์นี้แล้ว เขาต้องการทราบว่าเขาจะสามารถเก็บคนได้มากสุดกี่คน โดยไม่สนว่าใครเป็นใคร
ในสถานการณ์นี้ สนธยาสามารถเก็บได้มากสุดแค่ 3 คน โดยคุณเริ่มปลอมตัวเป็น ประยุทธ์ ก่อน
เพื่อไปเก็บประวิตร แล้วก็ปลอมตัวเป็นประวิตรเพื่อไปเก็บวิษณุ จากนั้นก็ปลอมตัวเป็นวิษณุเพื่อไปเก็บประยุทธ์
ซึ่งจะทำให้ปารีณาเป็นคนเดียวที่รอด หรือคุณสามารถเลือกเก็บปารีณาแทน เพื่อให้ประยุทธ์เหลือรอดก็ได้

\textbf{งานของคุณ} คือการหาว่าสนธยาสามารถเก็บคนได้มากสุดกี่คน โดยไม่สนว่าใครมีความสำคัญอย่างไร

\InputFile

\textbf{บรรทัดแรก} $Q$ แทนจำนวนคำถามที่คุณต้องตอบ

สำหรับแต่ละคำถามประกอบด้วย

บรรทัดแรกของคำถาม $N$ แทนจำนวนคน

สำหรับแต่ละคนประกอบด้วย $C$ และ $c_1, c_2, \cdots, c_C$ แทนจำนวนคนที่ไว้ใจ และคนที่ไว้ใจ

\OutputFile

ในแต่ละคำถาม มี 1 จำนวน แสดงคำตอบของคำถามนั้น

\Constraints

รับประกันว่าทุกคนจะมีคนที่ไว้ใจอย่างน้อย 1 คน

$Q \le 100$

\Subtasks

\textbf{ปัญหาย่อยทุกกลุ่มมีการจัดกลุ่ม}

\section*{ขอบเขตของปัญหาย่อยที่ 1-3}

$N \le 8$

\textbf{ปัญหาย่อยที่ 1} (3 คะแนน) ทุกคนรู้จักกันหมด

\textbf{ปัญหาย่อยที่ 2} (14 คะแนน) ทุกคนจะไม่ถูกไว้ใจจากคนที่ตัวเองไว้ใจ

\textbf{ปัญหาย่อยที่ 3} (22 คะแนน) ไม่มีเงื่อนไขเพิ่มเติม

\section*{ขอบเขตของปัญหาย่อยที่ 4-5}

$N \le 300$ อย่างมาก 20 คำถาม

สำหรับคำถามที่เหลือ $N \le 100$

\textbf{ปัญหาย่อยที่ 4} (28 คะแนน) $\sum{C}$ ในแต่ละคำถามไม่เกิน 1500

\textbf{ปัญหาย่อยที่ 5} (33 คะแนน) ไม่มีเงื่อนไขเพิ่มเติม

\Examples

\begin{example}
\exmp{
1
4
2 2 3
2 1 3
2 1 2
3 1 2 3
}{
3
}%
\end{example}

\end{problem}

\end{document}
